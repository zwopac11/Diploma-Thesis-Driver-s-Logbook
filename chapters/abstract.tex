\chapter*{Abstract}
\addcontentsline{toc}{chapter}{Abstract}
Every day, people go to work with their own, or the companys, car. To get taxes back from the state, the company has to carefully note down the driven trips. This procedure takes a lot of time,  because of the effort it takes to write this information down. Therefore the company suggested to automatically collect the needed data with an external device.\newline

Our Project consists of a device which is placed in a car. This hardware consists of an \gls{rpi2} and a \gls{gps} module. It tracks the distance and sums up the kilometers, which are driven by the car driver. This information gets inserted into a database. If there is no internet connection at the moment, the data is saved on the device and uploaded as soon as possible if there is an internet connection available. Afterwards, the user can log into his user account on our website or mobile application to see the tracks he or she drove. The user also can enter routes manual with the app or on the website. This can be separated in private and business usage of the car from the user. The possibility to edit the tracked ways is also given. It's also possible to view it on the phone or tablet via an mobile application.\newline

To sum up, we provide a device with a software on it, to create tracks via \gls{gps} data while driving. These are saved online and can be viewed on a website or mobile application.
