\chapter*{Abstract}
\addcontentsline{toc}{chapter}{Abstract}
Every day people go to work with their own or the company's car. To get taxes from the company for their driven kilometers, they always have to note down how many kilometers they drove for private and business purposes, which is a lot of work. But nevertheless it has to be done, because it gets added to the income. The private use of vehicles gets scheduled with 2\% of the acquisition cost. At vehicles with a lower \gls{co2} emission value the schedule is 1.5\% of the acquisition cost.
\newline \newline
Our Project consists of a device which is placed in a car. This hardware consists of an \gls{rpi2} and a \gls{gps} module. It tracks the distance and sum up the kilometers, which are driven by the car driver. This information gets inserted into a \gls{db}. If there is no internet connection at the moment, the data is saved on the device and uploaded as soon as possible if there is an internet connection available. Afterwards, the user can log into his user account on our website or mobile application to see the tracks he or she drove. The user also can enter routes manual with the app or on the website. This can be separated in private and business usage of the car from the user. The possibility to edit the tracked ways is also given. Its also possible to view it on the phone or tablet via an mobile application.
\newline \newline
To sum up, we provide a device with a software on it, to create tracks via \gls{gps} data while driving. These are saved online and can be viewed on a website or mobile application.