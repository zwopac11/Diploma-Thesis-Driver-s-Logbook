\chapter{Tools and Technologies}
\section{Java}
We decided on using JAVA, because it runs on every platform, is easy to implement and our partner company could use it on Android devices too. Also the fact that every member of the group got 5 years of programming experience in this language shows that it is the most fitting choice for us.
\section{TAPE API}
We implemented this \gls{api}, because of the problem with saving data into an offline file. This \gls{api} provides us a FileObjectQueue where we can put and remove our points. This FileObjectQueue is automatically saved in a file.
\section{GPS API}
This \gls{api} was rather hard to find because there are few possible ways of handling the \gls{gps} data from an \gls{rpi2} in JAVA. Nevertheless, we read through the excellent documentation and found ways to use this \gls{api} in an easy way.
\section{RASPBIAN}
As we tried out different things in linux on the Raspbian Wheezy during our NVSU lessons, it showed its potential. Another point why we used it, was that it was constructed to work perfectly with the \gls{rpi2}. Therefore, our first choice was this image for our \gls{rpi2}. Later we found out, that Raspbian Wheezy was not supported any more and so we had to switch to Raspbian Jessie Lite. Raspbian Jessie Lite is only a command line \gls{os}, so it is perfect for our project.
\section{PHP}
The usage of \gls{php} is based on \gls{rest}. It's functioning kind of like a server, which connects to the \gls{db}, uploads the \gls{gps} data and disconnects.
\section{BalsamiQ}
We used this tool to provide our wireframing plans. It was provided by our employer because they have made a very good experience when using it.
\section{GIT}
When working in a group, a good collaboration tool is needed. Our choice was GIT because we only made negative experiences.\newline
Unfortunately, we had problems at the beginning of our software development. Everytime we tried to "PULL" committed changes, we encountered a merge conflict. This conflict somehow followed us through the whole development phase.\newline
At the end, we also used GIT for LaTeX cooperation. Due to the usage of the GIT console, it was a very pleasant experience.
\section{MySQL}
Our employers standard as database technology was \gls{mysql}. Due to our experience we got in our DBI lessons, we encountered no problems.
\section{Netbeans}
The programming \gls{ide} of our choice was Netbeans. Over 4 years of experience in the same working environment payed off when working on a new, more complex set of problems on our own. These problems reached from logical errors to migrations of completely new programme structures. 
\section{Google Drive}
At the beginning and almost through the whole project, we used Google Drive for document management, time reporting and organization.\newline
The main Google products we have used were Google Docs and Google Tables.\newline
Collaborating was the main reason why we used especially this technologies. It worked trouble-free and had option to restore older versions of documents, which came in handy from time to time.
\section{Notepad++}

\section{LaXeT}

\section{Textmaker}

\section{draw.io}

\section{XAMPP}

\section{phpMyAdmin}

\section{Hangouts}
