\chapter{Tools and Technologies}
\pageauthor{Claudio Knapp}
\begin{wrapfigure}{r}{0.2\textwidth}
  \begin{center}
    \includegraphics[width=0.2\textwidth] {bilder/java}
  \end{center}
\end{wrapfigure}
\section{Java}
We decided on using JAVA, because it runs on every platform, is easy to implement and our partner company could use it on Android devices too. Also the fact that every member of the group got 5 years of programming experience in this language, shows that it is the most fitting choice for us.

\section{TAPE API}
We implemented this \gls{api}, because of the problem with saving data into an offline file. This \gls{api} provides us a FileObjectQueue where we can put and remove our points. This FileObjectQueue is automatically saved in a file.

\section{GPS API}
This \gls{api} was rather hard to find because there are few possible ways of handling the \gls{gps} data from an \gls{rpi2} in JAVA. Nevertheless, we read through the excellent documentation and found ways to use this \gls{api} in an easy way.

\newpage
\section{RASPBIAN}
\begin{wrapfigure}{r}{0.2\textwidth}
  \begin{center}
    \includegraphics[width=0.2\textwidth] {bilder/raspbian}
  \end{center}
\end{wrapfigure}
As we tried out different things in linux on the Raspbian Wheezy during our NVSU lessons, it showed its potential. Another point why we used it, was that it was constructed to work perfectly with the \gls{rpi2}. Therefore, our first choice was this image for our \gls{rpi2}. Later we found out, that Raspbian Wheezy was not supported any more and so we had to switch to Raspbian Jessie Lite. Raspbian Jessie Lite is only a command line \gls{os}, so it is perfect for our project.
\section{GIT}
\begin{wrapfigure}{r}{0.2\textwidth}
  \begin{center}
    \includegraphics[width=0.2\textwidth] {bilder/git}
  \end{center}
\end{wrapfigure}
When working in a group, a good collaboration tool is needed. Our choice was GIT because we only made negative experiences.

Unfortunately, we had problems at the beginning of our software development. Everytime we tried to "PULL" committed changes, we encountered a merge conflict. This conflict somehow followed us throughout the whole development phase.

At the end, we also used GIT for LaTeX cooperation. Due to the usage of the GIT console, it was a very pleasant experience.
\begin{wrapfigure}{r}{0.2\textwidth}
  \begin{center}
    \includegraphics[width=0.2\textwidth] {bilder/googledrive}
  \end{center}
\end{wrapfigure}
\section{Google Drive}
At the beginning and almost through the whole project, we used Google Drive for document management, time reporting and organization.
The main Google products we have used were Google Docs and Google Tables.

Collaborating was the main reason why we used especially this technologies. It worked trouble-free and had option to restore older versions of documents, which came in handy from time to time.
\clearpageauthor
\section{PHP}
\pageauthor{Helena Adam}
\begin{wrapfigure}{r}{0.2\textwidth}
  \begin{center}
    \includegraphics[width=0.2\textwidth] {bilder/php}
  \end{center}
\end{wrapfigure}
Originally it stood for Personal Home Page, but now for recursive backronym. It was designed for web development, but also used as programming language. 

The usage of \gls{php} is based on \gls{rest}. It's functioning kind of like a server, which connects to the \gls{db}, uploads the \gls{gps} data and disconnects.
\section{BalsamiQ}
\begin{wrapfigure}{r}{0.2\textwidth}
  \begin{center}
    \includegraphics[width=0.2\textwidth] {bilder/balsamiq}
  \end{center}
\end{wrapfigure}
We used this tool to provide our wireframing plans and mock ups, since it is a rapid wireframing tool. Mockups make it simple to generate ideas for designs and to pick the best one. It was recommended and provided by our employer because they have made a very good experience when using it. Also we made a great experience with it. It was easy to handle and our ideas could be implemented fast. Moreover, it was apparent if the design is right for the  user experience or rather not.
\begin{wrapfigure}{r}{0.2\textwidth}
  \begin{center}
    \includegraphics[width=0.2\textwidth] {bilder/MySQL}
  \end{center}
\end{wrapfigure}
\section{MySQL}
Our employers standard as database technology was \gls{mysql}. The open-source \gls{rdbms} is one of the most used client-server model \gls{rdbms}. It is a popular choice of a database for the use in a web application in general. Due to the experience we got in our DBI lessons, we encountered no problems.

\begin{wrapfigure}{r}{0.2\textwidth}
  \begin{center}
    \includegraphics[width=0.2\textwidth] {bilder/NetBeans}
  \end{center}
\end{wrapfigure}
\section{Netbeans}
The programming \gls{ide} of our choice was Netbeans. It is a software development platfrom written in Java and that is also the programming language we used at our work. Over 5 years of experience in the same working environment, it payed off when working on a new, more complex set of problems on our own. These problems reached from logical errors to the migration of completely new programme structures. 
\begin{wrapfigure}{r}{0.2\textwidth}
  \begin{center}
    \includegraphics[width=0.2\textwidth] {bilder/Notepad}
  \end{center}
\end{wrapfigure}
\section{Notepad++}
This program is a text editor for Windows that supports several languages. Since it is a portable app, you can do your work at all places as well. It provides features like tabbed documentation document interface or a FTP (File Transfer Protocol) browser. We used it to write the code for the web interface as we created \gls{php} files. Furthermore, we could look in the scripts of our database to check the DDL (Data Definition Language) or the DML (Data Manipulation Language) of our database.

\begin{wrapfigure}{r}{0.2\textwidth}
  \begin{center}
    \includegraphics[width=0.2\textwidth] {bilder/LaTeX}
  \end{center}
\end{wrapfigure}
\section{LaTeX and Texmaker}
We used the word processor and document markup language LaTex to make our diploma-thesis look more professional. In this program it is simpler to format the output and give it a nice shape compared to other document preparation systems. To create existing files of our diploma-thesis, we needed the tool Texmaker as well. This program is a cross-platform LaTex editor that integrates many tools needed to develop documents with LaTex.
%\begin{wrapfigure}{r}{0.2\textwidth}
  %\begin{center}
    %\includegraphics[width=0.2\textwidth] {bilder/Texmaker}
  %\end{center}
%\end{wrapfigure}
%\section{Texmaker}

\begin{wrapfigure}{r}{0.2\textwidth}
  \begin{center}
    \includegraphics[width=0.2\textwidth] {bilder/Draw_io}
  \end{center}
\end{wrapfigure}
\section{draw.io}
It is provided by Google to draw different kinds of \gls{uml} diagrams, flowcharts, process diagrams and so on. Because it is on drive, sharing and working on diagrams with the other users gets pretty simple and the day time does not matter anymore. So, as we were designing our database, we used it to create the \gls{erd} model for our database with all its entities and relations. Moreover, it got used to draw the Use Case and any other kind of diagrams we needed.

\begin{wrapfigure}{r}{0.2\textwidth}
  \begin{center}
    \includegraphics[width=0.2\textwidth] {bilder/XAMPP}
  \end{center}
\end{wrapfigure}
\section{XAMPP}
XAMPP makes it easy to create a local web server, since everything you need is setup for making a web server like a server application, database and \gls{php} as scripting language. It is an easy to install Apache distribution containing \gls{mysql}, \gls{php}, Perl and some other features. This means it works equally well on Linux, Mac and Windows. XAMPP is downloaded in a zip file. There is only little or no configuration required to make up the web server. It is regularly updated to incorporate the latest release of Apache, MariaDB, \gls{php} and Perl. Furthermore, it contains modules like OpenSSL, phpMyAdmin or WordPress.\newline
Over XAMPP it was possible to start pgAdmin and the SQL database for our diploma-thesis.

\begin{wrapfigure}{r}{0.2\textwidth}
  \begin{center}
    \includegraphics[width=0.2\textwidth] {bilder/phpMyAdmin}
  \end{center}
\end{wrapfigure}
\newpage
\section{phpMyAdmin}
To handle the administration of \gls{mysql} and our database with the use of a web browser, we used phpMyAdmin. It is an open source tool written in \gls{php} and can perform various tasks such as creating, modifying and deleting all around database. Furthermore, you can execute \gls{sql} statements with it and manage user and permissions.

\begin{wrapfigure}{r}{0.2\textwidth}
  \begin{center}
    \includegraphics[width=0.2\textwidth] {bilder/hangouts}
  \end{center}
\end{wrapfigure}
\section{Hangouts}
Because it is really hard to meet all the time in person during weekends or holidays, we used Hangouts to talk and discuss our diploma-thesis. It made the communication a lot easier for us and helped us to productively do our work.
\clearpageauthor