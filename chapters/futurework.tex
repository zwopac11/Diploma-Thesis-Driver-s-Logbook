\chapter{Future Work and Ideas}
Due to our limitation in time, we cannot infinitely improve our program. Therefore we sum up our future ideas on what could be implemented or improved. Of course, we only write about the things which came up in our minds, and therefore this is just a small list.
\newline \newline
\textbf{\textit{Security between \gls{rpi2} and REST service}} \newline
As we found out during the implementation process, REST supports an encryption process. After consulting the partner company, we decided on not using the encryption on the prototype.
\newline \newline
\textbf{\textit{Access via SSH over UMTS}}
\newline
It should be possible to access the \gls{rpi2} over the internet via SSH for remote maintenance.
\newline \newline
\textbf{\textit{Critical error reporting}}
\newline
If a severe problem happens, \gls{rpi2} should be able to report that problem so that it can be fixed.
\newline \newline
\textbf{\textit{Mobile Application/Webform}}
\newline
Using the wireframing of the webform and mobile devices, it is possible to create an access interface which can be accessed via a web browser or a mobile application.
\newline \newline
\textbf{\textit{Putting the GPS Points on the street}}
\newline
As shown in the Functional Testing part, the GPS data is rather inaccurate. Therefore our partner company has to interpolate and/or use the inaccuracy we add to the uploaded data. This task was not possible not accomplish in the given time, but could be implemented afterwards.

Compensating the inaccuracy will help to get an exact distance measurement and provide a better user experience.
\newline \newline
\textbf{\textit{Optimize Http overhead}}
\newline
If there are many points on the \gls{rpi2}  and the connection is established, the connection should only be closed after sending all the Points.
