\chapter{Future Work and Ideas}
Due to our limitation in time, we cannot infinitely improve our program. Therefore we sum up our future ideas on what could be implemented or improved. Of course, we only write about the things that came up to our minds, and therefore this is just a small list.
\paragraph{Security between RPi2 and REST service}\mbox{}\\
As we found out during the implementation process, \gls{rest} supports an encryption process. After consulting the partner company, we decided on not using the encryption on the prototype.
\paragraph{Access via SSH over UMTS}\mbox{}\\
It should be possible to access the \gls{rpi2} over the internet via \gls{ssh} for remote maintenance.
\paragraph{Critical error reporting}\mbox{}\\
If a severe problem happens, \gls{rpi2} should be able to report that problem so that it can be fixed.
\paragraph{Mobile Application/Webform}\mbox{}\\
Using the wireframing of the webform and mobile devices, it is possible to create an access interface that can be accessed via a web browser or a mobile application.
\paragraph{Putting the GPS Points on the street}\mbox{}\\
As shown in the Functional Testing part, the \gls{gps} data is rather inaccurate. Therefore our partner company has to interpolate and/or use the inaccuracy we add to the uploaded data. This task was not possible not accomplish in the given time, but could be implemented afterwards.

Compensating the inaccuracy will help to get an exact distance measurement and provide a better user experience.
\paragraph{Optimize Http overhead}\mbox{}\\
If there are many points on the \gls{rpi2}  and the connection is established, the connection should only be closed after sending all the Points.
