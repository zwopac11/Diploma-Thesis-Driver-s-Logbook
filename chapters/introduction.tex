\markboth{Introduction}{Introduction}  
\chapter*{Introduction}
\addcontentsline{toc}{chapter}{Introduction}
Being in contact with Sunlime IT Service before, we - Helena Adam, Claudio Knapp, Laura Rössl and Paul Zwölfer - got the the chance to do our diploma-thesis together with this company. As soon as everything was arranged, we had our first meeting with the CEO of Sunlime, DI Dominik Fuchshofer. He consigned us with our present work, namely with the driver’s logbook. His idea was it to disestablish the irritating noting down of where you went with the car and how much kilometers the distance had all together. After he gave and explained us the concept of the logbook, hours of hard work started where we tried to create the best out of it.\newline


One year has been past since we got the work and finally it has accomplished. Part of our project was the competitor analyse, the database, the wireframing, the project management and the prototype of the logbook hardware itself. All of these has been done to get an unique RPi2 prototype. Our software is running on this hardware and the collected data get saved in the from us created database. To have an overview of the process in this project, the project management was necessary. Anyway, these things are explained more in detail on the next pages. Our partner company was also not inactive. They managed everything around the data transfer, data acquisition and data design. Furthermore, they provided the web server for the portal and all technical parts we needed for the hardware prototype.\newline


Whenever we needed help at some technical points or something else, we could ask our supervisors and employer. They were always there to clear misunderstandings and supported us at our work during this period of time. However, also with the help of useful tutorials on the internet, we reached the end product of our diploma-thesis.\newline


Although, there were some hard times where we had struggles and sometimes a lack of knowledge in the area of GPS and other stuff, we managed to do it anyway and to reach our goal. But, it needs to be said that we had help of some incredible people. \newline


All in all, we have learned a lot of new things. For instance, how it is like to do a work in such a range in a team, or to deal with different situations and problems. Moreover, we were able to use new technologies and learn some new technical aspects.
\newpage
\section*{State of the Art}
Until now, the company has to track the driven distance by hand. They note down different parameters in an EXCEL sheet. There is a separate sheet for each month of the year. These parameters include the date, tachometer kilometers at the beginning and the end of the trip, how much kilometers are driven for the company or private use, the route they took, and the reason for driving.

This method is very time consuming and has to be done for every company related trip. Non company related trips must be excluded.
\subsection*{Our functionalities}
The \gls{rpi2} powers up when the driver starts up the car's ignition. After that, the autostart routine starts our software, including the process of collecting \gls{gps} data, saving and uploading it.

When the software started up, it checks if there is some \gls{gps} data from previous tracks to be uploaded. If that is the case, this data will be read from the save.jPoint file into the queue. After that, the save thread uploads a maximum of 50 points to the \gls{rest} service. 


Then the \gls{gps} data of the current track gets saved via the API thread into the queue and then synchronized with the save.jPoint file.

Reading and writing data simultaneously is possible due to the use of a blocking queue. This type of list manages the access of the list so that no errors occur.\newline
Successfully uploaded points are removed from the list and therefore from the synchronized backup file.

This runs through as long as the \gls{rpi2} is supplied with power.


\newpage
\section*{Role of the Company}
In this part of our diploma-thesis it is listed up what our employer, Sunlime IT Services, contributes to the work of the electronic logbook. Following areas are included:
\begin{itemize}
\item Hardware
\item Data transfer
\item Design requirements
\item Content requirements
\end{itemize}
The company provides the \gls{rpi2} with all the components like \gls{gps} Module or antenna to create a working hardware prototype for the logbook.

Secondly, at the data transfer, they will define when and over which portal the data has to be sent. Moreover, it should be apparent which user is transferring the data, so it can be dedicated online. It is also important to consider the user data when designing the \gls{db}. Furthermore, a feature list has to be defined and integrated.
The web server for the portal is also provided and maintained from the company.

An external service provider creates the design from the existing wireframing and Sunlime IT Services implements the technical aspects to it. The external service provider should also design a Logo for the product.

Thirdly, the company has to choose a “Mobile-first” approach to react to the display measures of each end device.

Then, the logbook should also work on smartphones. Therefore, Adobe PhoneGap will help to transact the program. The mobile application has to be designed one time only.

Finally, the company is responsible for the security on the servers. It should not be possible to have access on a server as third person. Passwords are only allowed to get saved encrypted.
\newpage
\section*{Goals}
The goal of our diploma-thesis is to build a working hardware prototype. This is recording the driven \gls{gps} data. A \gls{rpi2} with a \gls{gps} module has to be placed in a car. The prototype tracks \gls{gps} data and transfers it to the \gls{rest} service. \gls{rest} is responsible for uploading data into the \gls{db}. 

In case there is no internet connection, we store the \gls{gps} data locally. The program structure is described in our Software description section.

Information about the user and the tracks have to be stored in the \gls{db}. We also created and designed a \gls{erd} to illustrate the structure of the \gls{db}. Furthermore, the \gls{ddl} and the \gls{dml} resulted out of the \gls{erd}.

The wireframing is the Front-End representation of our diploma thesis. It displays the user interface on our web portal and mobile application. The result of it is a user-friendly design.

To get a unique product, the market needs to get analysed. Here, the competitors should get detected and compared in different attributes. For instance: costs, supported platforms, \gls{gps}, hardware etc. With this information, we create a diploma-thesis that differs from other products. Furthermore, other intended uses of the hardware prototype should get noted down.

Last but not least, the project management of the working packets has to be defined appropriate and documented.
