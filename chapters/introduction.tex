\newpage
\chapter*{Introduction}
\addcontentsline{toc}{chapter}{Introduction}
\section*{State of the Art}
Lalala state of the art text \gls{it} olalalal
\begin{minted}{Java}
System.out.println("lalala");1111
\end{minted}
lalallivrgbnwo \gls{it} jialwsg

\newpage
\section*{Role of the Company}
In this part of our diploma-thesis it is listed up what our employer, Sunlime IT Services, contributes to the work of the electronic logbook, which includes following areas:
\begin{itemize}
\item Hardware
\item Data transfer
\item Design requirements
\item Content requirements
\end{itemize}
The company provides the \gls{rpi2} with all the components like \gls{gps} Module or antenna to create a working hardware prototype for the logbook.

Secondly, at the data transfer, they will define when and over which portal the data has to be sent. Moreover, it should be apparent which user is transferring the data, so it can be dedicated online. It is also important to consider the user data when designing the \gls{db}. Furthermore, a feature list has to be defined and integrated.
The web server for the portal is also provided and maintained from the company.
\newline \newline
An external service provider creates the design from the existing wireframing and Sunlime IT Services implements the technical aspects to it. The external service provider should also design a Logo, which will be shown in the web portal.

Thirdly, the company has to choose a “Mobile-first” approach to react to the width of each end device.

Then, the logbook should also work on smartphones. Therefore, Adobe PhoneGap should help to transact the program. The mobile application has to be designed one time only.
\newline \newline
Finally, the company is responsible for the security on the servers. It should not be possible to have access on a server as third person. Passwords are only allowed to get saved encrypted.
\newpage
\section*{Goals}
The goal of our diploma-thesis is to build a working hardware prototype, which records \gls{gps} data. A \gls{rpi2} with a \gls{gps} module, has to be put into a car, to track \gls{gps} data and transfers it to a \gls{rest} interface. \gls{rest} is responsible for uploading data into the \gls{db}. Due to the fact that we need to upload the \gls{gps} data into our \gls{db}, we have to store the data locally on the \gls{rpi2} if there is no internet connection. The program structure is described in our Software description section.
\newline \newline
For our partner company, information about the user and for the user the tracks have to be stored in the \gls{db}. To illustrate the structure of the \gls{db}, an \gls{erd} got created and designed. Furthermore, the \gls{ddl} and the \gls{dml} resulted out of the \gls{erd}.
\newline \newline
The wireframing is the Front-End representation of our diploma thesis. It displays the user interface on our web portal and mobile application. A user-friendly design is the result of our wireframing.
\newline \newline
To get a unique product, the market needs to get analysed. Here, the competitors should get detected and compared in different attributes. For instance: costs, supported platforms, \gls{gps}, hardware etc. With this information, a diploma-thesis should be created which differs from other products. Furthermore, other intended uses of the hardware prototype should get noted down.
\newline \newline
Last but not least, the project management of the working packets has to be defined appropriate and documented.
\section*{Working Hours}
