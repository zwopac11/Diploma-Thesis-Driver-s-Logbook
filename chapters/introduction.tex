\markboth{Introduction}{Introduction}  
\chapter*{Introduction}
\addcontentsline{toc}{chapter}{Introduction}
\pageauthor{Laura Rössl}
Our project leader Helena Adam got in contact with the company Sunlime IT Services during previous internships. This is why we got the chance to work in cooperation with the company on our diploma-thesis. The \gls{ceo} of Sunlime, DI Dominik Fuchshofer, gave us actively support when necessary and was on hand for help and advice.
He had the idea to create an electronic driver’s logbook. If you use your private car for companys purpose, you have to note down driven trips for financial reasons. Using a company car for private and business trips, until now, you also had to note down the distances you drove.

One year has passed since we started working on our diploma thesis and finally we have completed. The main tasks of our project were the competitor analysis, the design of the database, the wireframing, the project management and the working prototype. The software we created runs on this prototype, collects \gls{gps} data and uploads it in our online \gls{db}. Our partner company provided the web server for the web portal and all technical components we needed. 

With the help of our supervising teachers and our employer, and useful tutorials on the internet, we were able to create a working prototype.
\clearpageauthor
\newpage
\section*{State of the Art}
\pageauthor{Claudio Knapp}
Until now, the company has to track the driven distance by hand. They note down different parameters in an EXCEL sheet. There is a separate sheet for each month of the year. These parameters include the date, tachometer kilometers at the beginning and the end of the trip, how much kilometers are driven for the companys or private use, the route they took, and the reason for driving.

This method is very time consuming and has to be done for every company related trip. Non company related trips must be excluded.
\subsection*{Our functionalities}
The \gls{rpi2} powers up when the driver starts up the cars ignition. After that, the autostart routine starts our software, including the process of collecting \gls{gps} data, saving and uploading it.

When the software started up, it checks if there is some \gls{gps} data from previous tracks to be uploaded. If that is the case, this data will be read from the save.jPoint file into the queue. After that, the save thread uploads a maximum of 50 points to the \gls{rest} service. 


Then the \gls{gps} data of the current track gets saved via the API thread into the queue and then synchronized with the save.jPoint file.

Reading and writing data simultaneously is possible due to the use of a blocking queue. This special type of queue manages the access of the list so that no errors occur.\newline
Successfully uploaded points are removed from the list and therefore from the synchronized backup file.

This runs through as long as the \gls{rpi2} is supplied with power.
\clearpageauthor

\newpage
\section*{Role of the Company}
\pageauthor{Helena Adam}
In this part of our diploma-thesis it is listed up what our employer, Sunlime IT Services, contributes to the work of the electronic logbook. Following areas are included:
\begin{itemize}
\item Hardware
\item Data transfer
\item Design requirements
\item Content requirements
\end{itemize}
The company provides the \gls{rpi2} with all the components like \gls{gps} Module or antenna to create a working hardware prototype for the logbook.

Secondly, at the data transfer, they will define when and over which portal the data has to be sent. Moreover, it should be apparent which user is transferring the data, so it can be dedicated online. It is also important to consider the user data when designing the \gls{db}. Furthermore, a feature list has to be defined and integrated.
The web server for the portal is also provided and maintained by the company.

An external service provider creates the design from the existing wireframing and Sunlime IT Services implements the technical aspects to it. The external service provider should also design a Logo for the product.

Thirdly, the company has to choose a “Mobile-first” approach to react to the display measures of each end device.

Then, the logbook should also work on smartphones. Therefore, Adobe PhoneGap will help to transact the program. The mobile application has to be designed one time only.

Finally, the company is responsible for the security on the servers. It should not be possible to have access on a server as third person. Passwords are only allowed to get saved encrypted.
\clearpageauthor
\newpage
\section*{Goals}
\pageauthor{Helena Adam, Claudio Knapp, Laura Rössl, Paul Zwölfer}
The goal of our diploma-thesis is to build a working hardware prototype. A \gls{rpi2} with a \gls{gps} module has to be placed in a car. The prototype tracks \gls{gps} data and transfers it to the \gls{rest} service. \gls{rest} is responsible for uploading data into the \gls{db}. 

In case there is no internet connection, we store the \gls{gps} data locally. The program structure is described in our Software description section.

Information about the user and the tracks have to be stored in the \gls{db}. We also created and designed an \gls{erd} to illustrate the structure of the \gls{db}. Furthermore, the \gls{ddl} and the \gls{dml} resulted out of the \gls{erd}.

The wireframing is the Front-End representation of our diploma thesis. It displays the user interface on our web portal and mobile application. The result of it is a user-friendly design.

To get a unique product, the market needs to get analysed. Here, the competitors should get detected and compared in different attributes. For instance: costs, supported platforms, \gls{gps}, hardware etc. With this information, we create a diploma-thesis that differs from other products. Furthermore, other possible uses of the hardware prototype should get noted down.

Last but not least, the project management of the work packages has to be defined appropriate and documented.
\clearpageauthor
