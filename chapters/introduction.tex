\newpage
\chapter*{Introduction}
\addcontentsline{toc}{chapter}{Introduction}
\section*{State of the Art}
Until now, the company has to track the driven distance by hand. They note down different parameters in an EXCEL sheet which is divided into the months of the year.These parameters include the date, tachometer kilometers at the beginning and the end of the trip, how much kilometers are driven for the company or private use, which route they took and the reason.

This method is very time consuming and has to happen at every driven trip which is for the company. 
\subsection*{What our functionality is:}
At first, the \gls{rpi2} starts up when the driver starts up the car's ignition. After that, the autostart routine starts our software and the the process of collecting \gls{gps} data, saving it and uploading it starts.

The software checks when started up if there is some \gls{gps} data from previous tracks to be uploaded. If that is the case, these data will be read from the save.jPoint from our save thread into the queue of all points and be uploaded to the \gls{rest} server in advance, so that the old track is entirely stored on the \gls{db}. 

After this step is completed, the \gls{gps} data of the current track gets saved via the API thread into the queue, which then saved for backup reasons into the save.jPoint file.

During this process, the points are read from the queue and will get uploaded to the \gls{rest} server. This is possible due to the use of a Blocking queue, which will manage the access of both threads so that no errors occur.\newline
The points which have been uploaded successfully are removed from the list and therefore from the synchronized backup file.

This runs through as long as the \gls{rpi2} is supplied with power.


\newpage
\section*{Role of the Company}
In this part of our diploma-thesis it is listed up what our employer, Sunlime IT Services, contributes to the work of the electronic logbook, which includes following areas:
\begin{itemize}
\item Hardware
\item Data transfer
\item Design requirements
\item Content requirements
\end{itemize}
The company provides the \gls{rpi2} with all the components like \gls{gps} Module or antenna to create a working hardware prototype for the logbook.

Secondly, at the data transfer, they will define when and over which portal the data has to be sent. Moreover, it should be apparent which user is transferring the data, so it can be dedicated online. It is also important to consider the user data when designing the \gls{db}. Furthermore, a feature list has to be defined and integrated.
The web server for the portal is also provided and maintained from the company.

An external service provider creates the design from the existing wireframing and Sunlime IT Services implements the technical aspects to it. The external service provider should also design a Logo, which will be shown in the web portal.

Thirdly, the company has to choose a “Mobile-first” approach to react to the width of each end device.

Then, the logbook should also work on smartphones. Therefore, Adobe PhoneGap should help to transact the program. The mobile application has to be designed one time only.

Finally, the company is responsible for the security on the servers. It should not be possible to have access on a server as third person. Passwords are only allowed to get saved encrypted.
\newpage
\section*{Goals}
The goal of our diploma-thesis is to build a working hardware prototype, which records \gls{gps} data. A \gls{rpi2} with a \gls{gps} module, has to be put into a car, to track \gls{gps} data and transfers it to a \gls{rest} interface. \gls{rest} is responsible for uploading data into the \gls{db}. Due to the fact that we need to upload the \gls{gps} data into our \gls{db}, we have to store the data locally on the \gls{rpi2} if there is no internet connection. The program structure is described in our Software description section.

For our partner company, information about the user and for the user the tracks have to be stored in the \gls{db}. To illustrate the structure of the \gls{db}, an \gls{erd} got created and designed. Furthermore, the \gls{ddl} and the \gls{dml} resulted out of the \gls{erd}.

The wireframing is the Front-End representation of our diploma thesis. It displays the user interface on our web portal and mobile application. A user-friendly design is the result of our wireframing.

To get a unique product, the market needs to get analysed. Here, the competitors should get detected and compared in different attributes. For instance: costs, supported platforms, \gls{gps}, hardware etc. With this information, a diploma-thesis should be created which differs from other products. Furthermore, other intended uses of the hardware prototype should get noted down.

Last but not least, the project management of the working packets has to be defined appropriate and documented.
\section*{Working Hours}
