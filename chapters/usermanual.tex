\chapter{User Manual}
\section{General}
First of all, thank you for buying our product. With your purchase on our product, you acquired a licence. Now you can register on our web portal and create your own account. Using your purchased licence, you can activate the product and connect it with your user account. You can also transfer the product with the related licence to another user account. Over the web portal, you can log in and view tracks, visualize them and eventually edit them.
\section{Installation Regulations}
The product has to be placed in the car and connected with the cigarette lighter receptacle to get power. When starting the engine of the vehicle, the device powers itself up and starts searching for a GPS signal.
\section{LEDs}
When you connect the product to a power source you will see some LEDs light up.
There are three LEDs. A red one on top of the \gls{rpi2}, which is for GPS module and two on the opposite side of the USB ports which are red and green.
When the red LED next to the green one is on, the \gls{rpi2} gets power.
The green LED on the \gls{rpi2} lights up when the device boots and prepares itself for tracking. When it is ready and it receives data it flickers quick.
And if the red LED on the \gls{rpi2} GPS module blinks once per second, it has a GPS signal
\section{Warnings}
\begin{itemize}
\item This product shall only be connected to an external power supply rated at 5V dc, and a minimum current of 500-700mA for model A and 700-1200mA for model B. Any external power supply used with the \gls{rpi2} shall comply with relevant regulations and standards applicable in the country of intended use. 
\item This product should not be overclocked as this may make certain components very hot. 
\item This product should be operated in a well ventilated environment and the case should not be covered. 
\item This product should be placed on a stable, flat, non-conductive surface in use and should not be contacted by conductive items. 
\item The connection of unapproved devices to the GPIO connector may affect compliance or result in damage to the unit and invalidate the warranty. 
\item All peripherals used with the \gls{rpi2} should comply with relevant standards for the country of use and be marked accordingly to ensure that safety and performance requirements are met. These articles include but are not limited to keyboards, monitors, and mice used in conjunction with the \gls{rpi2}.
\item Where peripherals are connected that do not include the cable or connector, the used cable or connector has to offer adequate insulation and operation in order that the requirements of the relevant performance and safety requirements are met.
\end{itemize}
\paragraph{To avoid malfunction or damage to your device please observe following:}
\begin{itemize}
\item Do not expose it to water, moisture or place on a conductive surface whilst in operation.
\item Do not expose it to heat from any source; the \gls{rpi2} is designed for reliable operation at normal ambient room temperatures.
\item Take care whilst handling to avoid mechanical or electrical damage to the printed circuit board and connectors.
\item Avoid handling the printed circuit board while it is powered. Only handle by the edges to minimise the risk of electrostatic discharge damage.
\item The \gls{rpi2} is not designed to be powered from a USB port of the other connected equipment. If this is attempted, it may malfunction.
\end{itemize}



